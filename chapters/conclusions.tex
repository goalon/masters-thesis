\chapter{Conclusions and future work}

\section{Answers to research questions}
Let me sum up the conclusions according to the gathered data.

Programmers work mostly during standard work hours with smaller portions of it done during the late evening. Differences in chronotypes between programmers of Python and JS/TS remain inconclusive.

Shorter breaks are much more frequent than longer ones, however the ones of length around 30 seconds are the most frequent and not the shortest ones. This may be due to the way how the plugin is constructed or really there is something tangible here about programmers' behavior.

During the span of the study, participants were observed to work only on a handful of distinct projects.

Differences in activity per days of the week are most visible when comparing normal weekdays with the weekend when the activity significantly drops. Performance calculations did not give conclusive insights into potential differences in performance between different days of the week.

\section{Key contributions}

\begin{itemize}
    \item Published open-source plugin to gather the programmers' data.
    \item Published open-source code to analyze the gathered programmers' data.
    \item Insights into the programming process of programmers regarding multiple variables.
\end{itemize}

\section{Future work}

The analysis and visualization of programming process is an extensive field of study. Because of that, it is possible to extend my study in many ways. Below I present my suggestions.

Gathering more data is an obvious choice for future work. That would make the results more believable and non-attributable as just some sort of a fluke. It would be important to check the programmers background to prepare a statistically appropriate group of study participants.

Extending the capabilities of the \texttt{MIMUW-MB-TT} plugin also seems like a straight-forward idea. Labeling the events as concrete programmers' actions (like switching a git branch) would require much more involvement from the participants (by requiring them to annotate the data points), but again would contribute to making the study more serious.

Multiple hypotheses may be constructed from my thesis. One of them that intrigues me is that in the figure \ref{fig:del_to_add} the ratio of deletions to additions seem to potentially be invariant to differences in data granularity. It may be interesting to challenge this claim.
