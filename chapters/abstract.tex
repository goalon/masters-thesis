\begin{abstract}
Similarly to another professions, programmers create their own working style. It can be characterized by different phases of the day, splitting tasks into a few chunks, taking breaks, or focusing on a few projects at the same time. In this thesis, I analyze if there exist links between programmers preferences and the spectre of their duties, for example, if it is true that front-end programmers work usually in the evenings. For this purpose, I conduct a research scheme on a sample of programmers with the help of a self-created software tool for gathering anonymized data (including the number of projects, time of activities and programming languages). The end result of this thesis comprises of the conclusions from the analysis of the aforementioned data with included visualizations. The tool used in the thesis is shared publicly for the willing actors to independently verify my end results by re-running the experiments or running new ones to gather new information, for instance, about potential improvements in programmers efficiency.
\end{abstract}
