\begin{thebibliography}{99}
\addcontentsline{toc}{chapter}{Bibliography}

\bibitem[1]{Rep02CircadianTiming} Steven M. Reppert, and David R. Weaver. 2002. \textit{Coordination of circadian timing in mammals}. Nature 418, 935–941, doi: \href{https://doi.org/10.1038/nature00965}{10.1038/nature00965}.

\bibitem[2]{Roe03Chronotypes} Till Roenneberg, Anna Wirz-Justice, and Martha Merrow. 2003. \textit{Life between Clocks: Daily Temporal Patterns of Human Chronotypes}. Journal of Biological Rhythms, 18(1), pp. 80–90, doi: \href{https://doi.org/10.1177/0748730402239679}{10.1177/0748730402239679}.

\bibitem[3]{Sto21ProgrammersDo} Dale Stokdyk. 2021. \textit{What Do Programmers Do, Anyway?} Southern New Hampshire University, url: \url{https://www.snhu.edu/about-us/newsroom/stem/what-do-programmers-do}.

\bibitem[4]{Fed21TimeTracking} Tony Fedorenko. 2021. \textit{Time Tracking in Software Development}. Mad Devs, url: \url{https://maddevs.io/customer-university/time-tracking-in-software-development/}.

\bibitem[5]{Gro21SDMHistory} Growin. 2021. \textit{A Brief History of Software Development Methodologies}. Growin Blog, url: \url{https://www.growin.com/blog/history-of-software-development-methodologies/}.

\bibitem[6]{Tho05} Richard C. Thomas, Amela Karahasanovic, and Gregor E. Kennedy. 2005. \textit{An investigation into keystroke latency metrics as an indicator of programming performance}. Proceedings of the 7th Australasian conference on Computing education - Volume 42 (ACE '05). Australian Computer Society, Inc., AUS, 127–134, doi: \href{https://dl.acm.org/doi/10.5555/1082424.1082440}{10.5555/1082424.1082440}.

\bibitem[7]{Lei16} Juho Leinonen, Krista Longi, Arto Klami, and Arto Vihavainen. 2016. \textit{Automatic Inference of Programming Performance and Experience from Typing Patterns}. Proceedings of the 47th ACM Technical Symposium on Computing Science Education (SIGCSE '16). Association for Computing Machinery, New York, NY, USA, 132–137, doi: \href{https://doi.org/10.1145/2839509.2844612}{10.1145/2839509.2844612}.

\bibitem[8]{Bar09} Hafez Barghouthi, \textit{How typing characteristics differ from one application to another}, Gjøvik University College, 2009

\bibitem[9]{Lei19Dissertation} Juho Leinonen. \textit{Keystroke Data in Programming Courses}. Ph.D. Dissertation, University of Helsinki, 2019

\bibitem[10]{Jos18} Aleix D. Josa, \textit{Identifying users using Keystroke Dynamics and contextual information}, University of Andorra, 2018

\bibitem[11]{Zav21MorningEvening} Albina Zavgorodniaia, Raj Shrestha, Juho Leinonen, Arto Hellas, and John Edwards. \textit{Morning or Evening? An Examination of Circadian Rhythms of CS1 Students}. 2021 IEEE/ACM 43rd International Conference on Software Engineering: Software Engineering Education and Training (ICSE-SEET), 2021, pp. 261-272, doi: \href{https://doi.org/10.1109/ICSE-SEET52601.2021.00036}{10.1109/ICSE-SEET52601.2021.00036}.

\bibitem[12]{Fei22DeveloperNames} Dror G. Feitelson, Ayelet Mizrahi, Nofar Noy, Aviad Ben Shabat, Or Eliyahu, and Roy Sheffer. \textit{How Developers Choose Names}. IEEE Transactions on Software Engineering, vol. 48, no. 1, pp. 37-52, 1 Jan. 2022, doi: \href{https://doi.org/10.1109/TSE.2020.2976920}{10.1109/TSE.2020.2976920}.

\bibitem[13]{Mat13PPV} Yoshiaki Matsuzawa, Ken Okada, and Sanshiro Sakai. 2013. \textit{Programming process visualizer: a proposal of the tool for students to observe their programming process}. Proceedings of the 18th ACM conference on Innovation and technology in computer science education (ITiCSE '13). Association for Computing Machinery, New York, NY, USA, 46–51, doi: \href{https://doi.org/10.1145/2462476.2462493}{10.1145/2462476.2462493}.

\bibitem[14]{Bie07FASTDash} Jacob T. Biehl, Mary Czerwinski, Greg Smith, and George G. Robertson. 2007. \textit{FASTDash: a visual dashboard for fostering awareness in software teams}. Proceedings of the SIGCHI Conference on Human Factors in Computing Systems (CHI '07). Association for Computing Machinery, New York, NY, USA, 1313–1322, doi: \href{https://doi.org/10.1145/1240624.1240823}{10.1145/1240624.1240823}.

\bibitem[15]{Lyu21TaskTracker} Elena Lyulina, Anastasiia Birillo, Vladimir Kovalenko, and Timofey Bryksin. 2021. \textit{TaskTracker-tool: A Toolkit for Tracking of Code Snapshots and Activity Data During Solution of Programming Tasks}. Proceedings of the 52nd ACM Technical Symposium on Computer Science Education (SIGCSE '21). Association for Computing Machinery, New York, NY, USA, 495–501, doi: \href{https://doi.org/10.1145/3408877.3432534}{10.1145/3408877.3432534}.

\bibitem[16]{Shr22CodeProcess} Raj Shrestha, Juho Leinonen, Arto Hellas, Petri Ihantola, and John Edwards. 2022. \textit{CodeProcess Charts: Visualizing the Process of Writing Code}. Australasian Computing Education Conference. Association for Computing Machinery, New York, NY, USA, 46–55, doi: \href{https://doi.org/10.1145/3511861.3511867}{10.1145/3511861.3511867}.

\bibitem[17]{Bal13SnapViz} Evan Balzuweit, and Jaime Spacco. 2013. \textit{SnapViz: visualizing programming assignment snapshots}. Proceedings of the 18th ACM conference on Innovation and technology in computer science education (ITiCSE '13). Association for Computing Machinery, New York, NY, USA, 350, doi: \href{https://doi.org/10.1145/2462476.2465615}{10.1145/2462476.2465615}.

\bibitem[18]{Nor08ClockIt} Cindy Norris, Frank Barry, James B. Fenwick Jr., Kathryn Reid, and Josh Rountree. 2008. \textit{ClockIt: collecting quantitative data on how beginning software developers really work}. Proceedings of the 13th annual conference on Innovation and technology in computer science education (ITiCSE '08). Association for Computing Machinery, New York, NY, USA, 37–41, doi: \href{https://doi.org/10.1145/1384271.1384284}{10.1145/1384271.1384284}.

\bibitem[19]{Cla18NightWeekend} Maëlick Claes, Mika V. Mäntylä, Miikka Kuutila, and Bram Adams. 2018. \textit{Do programmers work at night or during the weekend?} Proceedings of the 40th International Conference on Software Engineering (ICSE '18). Association for Computing Machinery, New York, NY, USA, 705–715, doi: \href{https://doi.org/10.1145/3180155.3180193}{10.1145/3180155.3180193}.

\bibitem[20]{Sta22StackOverflow} Stack Exchange Inc. 2022. \textit{Stack Overflow Main Page}, Stack Overflow, url: \url{https://stackoverflow.com/}.

\bibitem[21]{Rob17LangsUsedAtNight} David Robinson. 2017. \textit{What Programming Languages Are Used Late at Night?} Stack Overflow Blog, url: \url{https://stackoverflow.blog/2017/04/19/programming-languages-used-late-night/}.

\bibitem[22]{Sil17LangsUsedOnWeekends} Julia Silge. 2017. \textit{What Programming Languages Are Used Most on Weekends?} Stack Overflow Blog, url: \url{https://stackoverflow.blog/2017/02/07/what-programming-languages-weekends/}.

\bibitem[23]{Nov13SEV} Renato Lima Novais, André Torres, Thiago Souto Mendes, Manoel Mendonça, and Nico Zazworka. 2013. \textit{Software evolution visualization: A systematic mapping study}. Information and Software Technology, Volume 55, Issue 11, 2013, Pages 1860-1883, ISSN 0950-5849, doi: \href{https://doi.org/10.1016/j.infsof.2013.05.008}{10.1016/j.infsof.2013.05.008}.

\bibitem[24]{Mat16SoftwareVisualizationReview} Anna-Liisa Mattila, Petri Ihantola, Terhi Kilamo, Antti Luoto, Mikko Nurminen, and Heli Väätäjä. 2016. \textit{Software visualization today: systematic literature review}. Proceedings of the 20th International Academic Mindtrek Conference (AcademicMindtrek '16). Association for Computing Machinery, New York, NY, USA, 262–271, doi: \href{https://doi.org/10.1145/2994310.2994327}{10.1145/2994310.2994327}.

\bibitem[25]{Vih14CodeSnapshotGranularity} Arto Vihavainen, Matti Luukkainen, and Petri Ihantola. 2014. \textit{Analysis of source code snapshot granularity levels}. Proceedings of the 15th Annual Conference on Information technology education (SIGITE '14). Association for Computing Machinery, New York, NY, USA, 21–26, doi: \href{https://doi.org/10.1145/2656450.2656473}{10.1145/2656450.2656473}.

\bibitem[26]{Jad06NoviceCompilationBehaviour} Matthew C. Jadud. 2006. \textit{Methods and tools for exploring novice compilation behaviour}. Proceedings of the second international workshop on Computing education research (ICER '06). Association for Computing Machinery, New York, NY, USA, 73–84, doi: \href{https://doi.org/10.1145/1151588.1151600}{10.1145/1151588.1151600}.

% todo check
\bibitem[27]{Hum96PSS} W. S. Humphrey, \textit{Using a defined and measured Personal Software Process}. IEEE Software, vol. 13, no. 3, pp. 77-88, May 1996, doi: 10.1109/52.493023.

% todo review
\bibitem[28]{DieSoftViz} Stephan Diehl. 2007. \textit{Software visualization: visualizing the structure, behaviour, and evolution of software}. Springer Science \& Business Media.

% todo review
\bibitem[29]{Microsoft} https://news.microsoft.com/facts-about-microsoft/

% todo review
\bibitem[30]{Haskell} https://www.haskell.org/

% todo review
\bibitem[31]{Firebase} https://firebase.google.com/

% todo review
\bibitem[32]{Unity} https://unity.com/

% todo review
\bibitem[33]{Con92SDM} Danny T. Connors. 1992. \textit{Software development methodologies and traditional and modern information systems}. SIGSOFT Softw. Eng. Notes 17, 2 (April 1992), 43–49. https://doi.org/10.1145/130840.130843

% todo review
\bibitem[34]{Put19FourChronotypes} Arcady A. Putilov, Nele Marcoen, Daniel Neu, Nathalie Pattyn, Olivier Mairesse. 2019. \textit{There is more to chronotypes than evening and morning types: Results of a large-scale community survey provide evidence for high prevalence of two further types}. Personality and Individual Differences, Volume 148, 2019, Pages 77-84, ISSN 0191-8869, https://doi.org/10.1016/j.paid.2019.05.017.

% todo review
\bibitem[35]{Trello} https://trello.com/

% todo review
\bibitem[36]{Jira} https://www.atlassian.com/software/jira

% todo review
\bibitem[37]{YouTrack} https://www.jetbrains.com/youtrack/

% todo review
\bibitem[38]{Timely} https://timelyapp.com/

% todo review
\bibitem[39]{WakaTime} https://wakatime.com/

% todo review
\bibitem[40]{CodeTime} https://wakatime.com/codetime

% todo review
\bibitem[41]{Tempo} https://www.tempo.io/

\end{thebibliography}

% https://www.jetbrains.com/opensource/idea/
