\section{Tracking programmers time in the IT industry}

In many areas of the IT industry, programmers salaries are the most significant expense and not the computation time (electricity bills, server maintenance etc.). In such a case, effective management of these human resources becomes a crucial task for keeping finances of a company in check. That is why it is essential to know exactly how many hours need to be spent for each programmer and on which tasks. For this purpose many tools were created to generate relevant reports for the employer's oversight on the company and for the employees themselves to motivate them to earnest effort.

The tools like \href{https://trello.com/}{Trello}, \href{https://www.atlassian.com/software/jira}{Jira}, \href{https://www.jetbrains.com/youtrack/}{YouTrack} allow for manual time tracking. The programmers are required to manually mark their progress. They are tightly coupled with software development methodologies (established ways of navigation through each phase of the software development \cite{Con92SDM}). Handling agile frameworks such as Kanban and Scrum is especially important for these tools.

Nevertheless, there are also tools which help with automated time tracking. Such tools include \href{https://timelyapp.com/}{Timely} and \href{https://www.tempo.io/automated-time-tracking}{Tempo} products. They track user actions and evaluate them using different techniques (including those based on machine learning) to label them accordingly. This way the users can save the time on filling out the reports, especially thanks to built-in integrations with Jira etc. Another interesting solution is to integrate dashboards directly into an IDE. Notable examples of that approach include \href{https://wakatime.com/}{WakaTime} and \href{https://www.software.com/product/code-time}{CodeTime}. The visualizations and statistics that these tools conveniently generate for the IDE users are comparable to those available in my tool. However, one of my main premises was to ensure the availability and security of the gathered data adhering exactly to my specification. Thus, I have decided to built my own tool according to the aforementioned principles.

The agile software development methodologies mentioned above (Kanban and Scrum) utilize special dashboards with clearly described tasks for programmers. They represent an interesting way of visualizing activities of developers. My thesis does not cover detailed labeling of activities which is required by those techniques. Nevertheless, it is a concept worth noting for potential further research. The emergence of the automated time tracking tools (e.g. Timely and Tempo products) show that the IT industry is attentive to the developments in the area of sophisticated analyzing and visualizing of programmers' activities. I hope my thesis also gives some insight into this matter.
