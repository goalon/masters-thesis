% I mean programmers data - that is mostly obtained keystrokes - I don't really care about surveys and I need to underline it potentially in the introduction simply
\section{Usage of programmers data in research}

% Industry research
% JetBrains research
% University research - Juho Leinonen

% \textcolor{red}{TODO}
Analysis of gathered programmers data is a vibrant area of research. I would like to make a quick overview of such research, especially this most relevant to my thesis.
% \textcolor{red}{TODO: mention literature review studies and in visualization as well!}

% How did he actually obtain this data?
% Compare with my thesis (e.g. the number of participants in data gathering)
Zavgorodniaia et. al. \cite{Zav21MorningEvening} collected keystroke data from two groups of introductory programming students (European and American) using a custom IDE. This data allowed the research team to search for the activity patterns associated with distinct chronotypes. They identified four chronotypes isolated by Putilov et. al. \cite{Put19FourChronotypes} that is morning, napper, afternoon, and evening. The difference in scores was the most noticeable within the US context, where morning persons achieved the highest exam scores. Comparing the contexts, it was observed that European participants usually started their work earlier than their American counterparts. Researchers stipulated that the coincidence of their results with the prior published research does not proof that students’ circadian rhythms and diurnal preferences are the sole reason for their activity patterns. Social factors like rigid schedules or external duties might have impacted the conclusions since the surveys for such factors were not taken as part of the experiments.

Claes et. al. \cite{Cla18NightWeekend} examined programmers work patterns by analyzing version control commits from multiple software projects. The researchers found that typically developers follow the standard circadian rhythm where effort is concentrated in the morning and in the afternoon. Unsurprisingly, they observed characteristic dips in activity around the lunch time (11 or 12 o'clock depending on the project) and much bigger ones during the weekends. The myth of programmers being night owls was not confirmed since according to the data the median 8-hour working day is placed between 10 and 18 o'clock. In addition, the data showed that two thirds of developers work primarily within the office hours. The researchers found some evidence that activity outside the normal office hours is correlated with non-paid (hobby) work.

% Automatic inference of typing patters

Robinson \cite{Rob17LangsUsedAtNight} examined in his article the data from Stack Overflow about visits to the site. Stack Overflow \cite{Sta22StackOverflow} is a popular platform where developers ask and answer technical questions. The author of the article decided to inspect developers programming behavior assuming a strong link between Stack Overflow and programming activity. Not surprisingly, the peak activity was observed during the weekdays within the standard 9-to-5 time period (adjusted for local time zones). Moreover, a dip in traffic was observable at 12 p.m. coinciding with a typical lunch hour. On weekends such a dip did not exist. Across several programming languages there were noticed slight differences in hourly trends, that is some of the languages were e.g. associated with statistically earlier or later times of activity. The author also highlighted the technologies that were relatively most used within or outside the typical working hours (9-to-5). Multiple Microsoft (a major IT company \cite{Microsoft}) products demonstrated high percentages of usage within the working hours, while Haskell (a functional programming language \cite{Haskell}), Firebase (an application development platform \cite{Firebase}) and Unity (a game engine \cite{Unity}) outside of them. In addition to that, there was analyzed usage in the morning and the afternoon, and found a general correlation between usage within working hours and in the morning. At the end, it was discovered that cities known for remote work exhibit a bigger number of late chronotypes comparing to traditional software/business centers with locally employed programmers.
% \textcolor{red}{TODO Weird!}

% \textcolor{red}{TODO weird - rewrite}
Silge \cite{Sil17LangsUsedOnWeekends} explored the diversity of questions that are posted on weekdays and weekends. She noticed that some languages become more popular during weekends (with regards to the relative frequency on weekdays vs on the weekend). There are also trends of diversion in these statistics throughout time which may suggest maturing of certain technologies (moving activity from weekends to weekdays) or going more mainstream with hobbyists (moving activity from weekdays to weekends).
