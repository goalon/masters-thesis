\section{Data description}\label{sec:data_description}

The data is stored in the file system in the \texttt{JSON Lines} (\texttt{.jsonl}) \cite{JSONLines} format. It was popularized in recent years by OpenAI (a popular AI/ML company) in its GPT-3 API \cite{OpenAIGPT3}. Essentially files in this format are text files where each line (seperated by newline "\verb"\n"" characters) constitute a valid JSON value (that is why this format is sometimes also called \texttt{newline-delimited JSON}). In the particular case of my tool each of these JSON lines has a special designated type \ref{fig:event_type}. As an example I present below a sample of the gathered data \ref{fig:data_file_example}.

\begin{figure}[htbp]
    \centering
    \begin{verbatim}
{
  additions:                 INTEGER
  deletions:                 INTEGER
  actions:                   INTEGER
  start:                     STRING      // ISO 8601 date-time format
  end:                       STRING      // ISO 8601 date-time format
  workspaceNameHash:         STRING      // SHA-256 string or "null"
  languages: {               DICTIONARY
    [languageId: STRING]: {  DICTIONARY
      additions:             INTEGER
      deletions:             INTEGER
      actions:               INTEGER
    }
  }
}
\end{verbatim}

    \caption{The data type}
    \label{fig:event_type}
\end{figure}

Let me describe each of the fields of this structure:
\begin{description}
    \item[\texttt{additions}] say how many characters (chars) were added to the files opened in the editor within the time-frame determined by the fields \texttt{start} and \texttt{end};
    \item[\texttt{deletions}] say analogously to the \texttt{additions} field about the deleted chars;
    \item[\texttt{actions}] count how many actions were done on each file opened in the editor (one action may add or delete multiple chars);
    \item[\texttt{start}] means the start of the interval during which the record of the data was gathered;
    \item[\texttt{end}] is analogously to the \texttt{start} field the end of the interval;
    \item[\texttt{workspaceNameHash}] is the SHA256 \cite{NIST02SHS} hash of the workspace name in VSCode. It is done to secure the privacy of the research participants in the well established way (not to leak these names), but simultaneously to still be able to differentiate between the project names. Thanks to this, I am able to use this field to calculate the number of distinct projects of each user. If the workspace name does not exist, the value of \texttt{workspaceNameHash} is \texttt{null};
    \item[\texttt{languages}] is a dictionary that stores data separately for each programming language. It basically splits the fields \texttt{additions}, \texttt{deletions} and \texttt{actions} into contributions from each language. To compute this data, I have used the feature of VS Code that automatically recognizes a programming language from the file extension \cite{Mic22Langs}.
\end{description}

\begin{figure}[htbp]
    \scriptsize{
\begin{verbatim}
{"additions":49,"deletions":2,"actions":59,
"workspaceNameHash":"47ba341f442134c53c9cd42d8e4ab6d627ab7ca632ffdf60a432c7c6e2df9544",
"languages":{"javascript":{"additions":49,"deletions":2,"actions":59}},
"start":"2022-10-17T14:15:04.381+02:00","end":"2022-10-17T14:15:30.636+02:00"}

{"additions":105,"deletions":11,"actions":30,
"workspaceNameHash":"47ba341f442134c53c9cd42d8e4ab6d627ab7ca632ffdf60a432c7c6e2df9544",
"languages":{"javascript":{"additions":105,"deletions":11,"actions":30}},
"start":"2022-10-17T14:16:26.214+02:00","end":"2022-10-17T14:16:43.514+02:00"}

{"additions":12,"deletions":6,"actions":20,
"workspaceNameHash":"47ba341f442134c53c9cd42d8e4ab6d627ab7ca632ffdf60a432c7c6e2df9544",
"languages":{"javascript":{"additions":12,"deletions":6,"actions":20}},
"start":"2022-10-17T14:16:59.027+02:00","end":"2022-10-17T14:17:28.828+02:00"}
\end{verbatim}
}

    \caption{Fragment of an example data file with 3 data points (normally there is one data point per line, but here the lines are wrapped to fit them on the page)}
    \label{fig:data_file_example}
\end{figure}
