\chapter{Background}\label{r:background}

\section{Basic concepts}

\begin{definition}\label{circadian_rhythm}
\textcolor{red}{TODO REF!!!}
A \emph{circadian rhythm} is a cycle of internal oscillations in
nearly all physiological activities generated by the molecular
circadian clock and has a period of approximately 24 hours.
\end{definition}

%TODO LABELS

\begin{definition}
\textcolor{red}{TODO REF!!!}
\emph{Chronotype} is a person’s natural inclination toward
activity at certain periods of the day and depends on a circadian rhythm for synchronization.
\end{definition}

\begin{definition}
A \emph{diurnal rhythm} ...
\end{definition}

% maddevs.io
\begin{definition}
\textcolor{red}{TODO REF!!!}
\emph{Time tracking} refers to how a business logs time for employees. It shows how much time an employee has spent on a task.
\end{definition}

\begin{definition}
\emph{Keystroke dynamics} refers to ...
\end{definition}

My work is focused on analyzing chronotypes of different programmers and finding relations between chronotypes and different \textcolor{red}{(TODO phrasing)} factors like programming languages used.

% The definitions above will be widely used throughout the thesis.

\section{Tracking programmers time in the IT industry}

In many areas of the IT industry, programmers salaries are the most significant expense
and not the compute time. In such a case, effective management of these human resources
becomes a crucial task for keeping finances of a company clean.

% I mean programmers data - that is mostly obtained keystrokes - I don't really care about surveys and I need to underline it potentially in the introduction simply
\section{Usage of programmers data in research}

% Industry research
% JetBrains research
% University research - Juho Leinonen

\subsection{Enterprise context}

\subsection{Educational context}

% TODO it's not really latest
% How did he actually obtain this data?
% Compare with my thesis
\textcolor{red}{TODO REF PLAG!!!}
In his latest research Leinonen et. al. \cite{lei21} looked at analyzed evidence for the existence of chronotypes using keystroke data collected from introductory programming students. He iden-
tified four chronotypes similar to those discussed in the
literature [23]: morning, napper, afternoon, and evening.
We observed noticeable differences
in the exam scores and project scores within the US context,
where those active in the morning performed the best in the
exam.
We observed differences between the contexts inwhen the students started to work on their projects and noticed
that, in general, students in the European context tended to
start their work earlier. We hypothesized that this could be
due to two factors.
In this article we have inferred behavior from observed
keystrokes and while our conclusions are in line with prior
theoretical and empirical research, we cannot for certain
say whether our observations stem from students’ circadian
rhythms and diurnal preferences, or whether there are other
factors at play.

% TODO reconsider wording
\section{Available tools for time tracking}
\subsection{Used by the industry}
\subsection{Used by researchers}
% Timely - for billing not for keystrokes tracking
% no keystroke monitoring - mentioned as a feature (no "creepy" surveillance tactics)

% and why I'm not using them - that is those tools above

% Difference in convenience

% TODO reconsider wording and structure
\section{Advances in visualization strategies}
\subsection{Enterprise context}
\subsection{Educational context}

\textcolor{red}{TODO REF PLAG!!!}
In this work, we presented CodeProcess which is a novel tool for vi-
sualizing the programming process (example visualizations shown
in Figure 2). The tool utilizes keystroke data to show in which order
different parts of the source code were developed. In addition, we
conducted a pilot think-aloud study to evaluate whether computing
instructors can leverage the visualizations for pedagogical insights.
Our aim in developing the tool was to provide instructors with
easy-to-understand visualizations that tell something about the
process a student took to arrive at their solutions at a glance. We
hypothesized that instructors could use the tool – for example – to
augment assessment; to determine whether students are solving a
programming problem in a top-down or a bottom-up manner; that
the tool could be used to identify cases of plagiarism; and that the
visualization could also indicate whether a student is struggling. Ad-
ditionally, the tool could be used to visualize the programming pro-
cess to the student themselves for reflection, or show students their
peers’ processes to allow students to see other solution approaches
and problems other students might have had when programming.
These analyses could be enhanced through first identifying specific
cases – or stereotypical cases – from the data using, say, machine
learning methodologies.
Lastly, we suggest that the
tool could also be used in the professional context for code reviews
and allow professional programmers to reflect on their process.
