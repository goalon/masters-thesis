\chapter{Introduction}

Programmers job focuses on developing computer software through all phases including design, testing and maintenance. These processes are executed according to the principles of performance, reliability and security \cite{Sto21ProgrammersDo}. With the industry maturing, upholding these rules in ever-expanding projects have become crucial to fulfill growing needs stated i.a. in software level agreements (SLAs). One of the answers to that have been the agile software development \cite{Gro21SDMHistory}, where actions like joining regularly scheduled team meetings, contacting clients or attending conferences have been universalized. Nevertheless, writing code has remained a key part of programmers job for the majority of them. In my study, I focus on analyzing and visualizing the exact process of writing software in an integrated development environment (IDE), largely omitting the other aforementioned aspects of being a programmer.

Research in the relevant areas differs depending on the context that is industrial or educational. Tracking of programmers time happens to be an essential part of the job in the IT industry. The reports provided by programmers are the basis for calculating a salary or even deciding the future of an employee by the management. For many programmers it may be hard to determine which activities constituted billing time or if different activities should be treated differently. In my own experience, hard to fix bugs or complicated issues solving which took many trials and errors happened to be particularly hard when it comes to calculating the exact amount of billing hours it took me and how to prove this reported effort to my employer. Those obstacles constituted my initial desire to explore the domain of programming process analysis and visualization. Nowadays there exist tools that measure the time automatically and compile the reports with almost no additional human effort.

In the industry there is less focus on the detailed analysis of the process of writing the code and more on the activities done during the whole workday. Process of writing the code is more of a concern for educational research and IDE development companies like JetBrains who want to know how users interact with their products and how to make them more ergonomic. In educational research people try to evaluate students based on their typing behavior. Apart from that, it's possible to authenticate users based on their typing characteristics.

% It was then I came up with researching this area a bit further. My idea was to track programmers activities even before they create any meaningful output. Of course, you could argue that the paradigms of continuous integration already solve this problem, however at the heart of my concerns lied the relatively high granularity of the data, which would be practically unachievable just by utilizing common paradigms (you do not imagine a programmer committing her/his work to git every minute or even more frequently without a high nosedive of her/his productivity).

The thesis contains 5 chapters and appendices. In the chapter \ref{r:background} there are studied basic concepts and related works both in analysis and in visualization techniques. The chapter \ref{ch:methodology} describes the tool used to gather and analyse the data. In the chapter \ref{ch:results} the results are presented and in the next one \ref{ch:discussion} they are discussed. The last two chapters conclude the whole thesis with suggestions for future work \ref{ch:conclusions} and provide links to the additional published materials \ref{ch:publishing}. I clarify that the pronoun “I” used in this thesis refers to its author.

% Say what it is about and what not. Say what data you use and which not. Say what to expect.

% Say that by activities you mostly mean just keystrokes.

% It's important to mention here that I am not interested in resources of data such as surveys, and only the factual data sources as there exists evidence that the programmers perception is inconsistent with the real data. Reference here.

% Mention that educational approaches can be used in the industry in the future.
